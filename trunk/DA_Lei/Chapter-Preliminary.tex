\chapter{Preliminary}\label{pre}

In this chapter we firstly formalize the keyword search problem on data web in
section~\ref{pre:ksdw}, where various models used throughout this thesis are discussed and defined.
Section~\ref{pre:routing} introduces the problem of keyword query routing tackled by this thesis.

\section{Problem Definition of Keyword Search on the Data Web}\label{pre:ksdw}

As mentioned in section~\ref{intro:motivation}, in the data web search setting, answering keyword
queries might involve multiple data sources. In addition, it is assumed that these data sources are
not independent but interlinked by mappings. We elaborate on the problem of keyword search
on data web in what follows. We will begin with a model for underlying data, and continue with the
query and answer models followed by the scoring model. At last we will introduce the problem of
keyword query routing for data web search based on the defined models.

\subsection{Data, Query and Answer Models}

\paragraph{Data Model}
We consider the data web as a set of interrelated web data sources, each of them identified using a
data source identifier. We use a graph-based data model to characterize individual web data sources.
Since the structured data on the web varies greatly in their forms, such as XML data, relational data
and RDF\footnote{\url{http://www.w3.org/RDF/}}, a framework for (Web) resource description
standardized by the W3C, we will present a general data model that is sufficiently to capture all of
them.

\begin{definition}\label{def:datagraph}
A \emph{data graph} $G$ is a tuple $(V,L,E)$ where
\begin{itemize}
  \item $V$ is a finite set of \emph{vertices}. Thereby, $V$ is
  conceived as the disjoint union $V_E\uplus
  V_C\uplus V_V$ with E-vertices $V_E$
  (representing entities), C-vertices $V_C$
  (classes), and V-vertices $V_V$ (data values).
  \item $L$ is a finite set of \emph{edge
  labels}, subdivided by $L = L_R \uplus L_A \uplus
  \{type, subclass\}$, where $L_R$ represents
  inter-entity edges and $L_A$ stands for
  entity-attribute assignments. \item $E$ is a
  finite set of \emph{edges} of the form
  $e(v_1,v_2)$ with $v_1,v_2\in V$ and $e\in L$.
  Moreover, the following restrictions apply:
  \begin{itemize}
    \item $e \in L_R$ if and only if $v_1, v_2 \in V_E$,
    \item $e \in L_A$ if and only if $v_1 \in V_E$ and $v_2 \in V_V$,
    \item $e = \mathit{type}$ if and only if $v_1 \in V_E$ and $v_2 \in V_C$, and
    \item $e = \mathit{subclass}$ if and only if $v_1, v_2 \in V_C$.
  \end{itemize}
\end{itemize}
\end{definition}

To interrelate the elements of individual data sources, our data
model is extended with mappings:

\begin{definition} A \emph{mapping} $M$ is set of mapping assertions representing approximate
correspondences between graph elements. Specifically, mapping assertions in $M$ are of the form
$m(v_1,v_2, s)$ where $v_1,v_2 \in V$ are graph vertices  and $s \in [0,1]$ is a score denoting a
confidence value associated with the mapping.
\end{definition}

Data sources together with mappings relating them form a \emph{data web} defined as an {\em integrated data graph}:

\begin{definition} An \emph{integrated data graph} $g_{ID}$ is a tuple $(G_D,M_D)$, where $G_D$ is a
finite set of data graphs and $M_D$ is a set of approximate correspondences between data elements
called \emph{individual mappings}.
\end{definition}

While edges of a particular graph are called
\emph{intra-data-source edges}, edges representing mappings
between elements of different data graphs will be referred to as
\emph{inter-data-sources edges}.

\paragraph{Query Model}

\paragraph{Answer Model}

\subsection{Relevance Scoring Model}

%content

\section{Problem Definition of Keyword Query Routing}\label{pre:routing}

As discussed in section~\ref{intro:motivation}, identifying the most relevant data sources to user
keyword queries and directing the keywords to appropriate sources will be a key to our data web
search setting.



