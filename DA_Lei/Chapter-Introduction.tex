\chapter{Introduction}\label{intro}\index{introduction}

\section{Motivation}\label{intro:motivation}
In recent years, with the advent and popularity of a vast amount of structured data available on the
Web, which is ever increasing rapidly, the Web as a global information space is no longer only a Web
of documents, but a Web of data - {\em data web}. Currently, there are billions of triples stored in
web data sources of different domains. These data sources become more tightly interrelated as the
number of links in form of mappings is also growing. This development of a data web offer
opportunities for addressing complex information needs. Besides searching for documents, the user can
more specifically ask for the precise information using more expressive queries according to
structured query languages. However, structured query language is too complicated for casual users,
which hinders them in expressing their information need. As keyword search has been proven to be
useful for document retrieval on the web, it is much easier to handle than structured query languages
for casual users. The problem of keyword search on the data web has been studied in our previous work
SearchWebDB \citep{TranTR}.

Unfortunately, the scale and the diversity of the data web brings about some new challenges. In this
data web search setting, where a result containing all query keywords may span over some distinct
data sources linked by mappings, we face the challenge of searching an often large space of potential
query results to quickly find the top few results for a user query. Searching this large space is
much harder than in a single data source because it grows exponentially with the number of data
sources and their associated mappings. This significantly increased search space render single data
source exploration strategies inefficient for the integrated setting which involves a large number of
data sources. In most of the approaches that aim at the problem of keyword search of the structured
data in a single data source, the systems first map the keywords to the keyword elements containing
any query keyword, then start the exploration from all keyword elements to find the results. In data
web search setting, it is not feasible because the cost of exploration is too high. Futhermore, in
many cases, even more keyword elements and corresponding data sources do not participate in any
result such that computations are largely wasted.

Therefore, we would ideally like a solution where given an arbitrary user keyword query, we identify
just the right data sources that are likely to have the results and direct the keywords in the query
to only the relevant sources. The problem of identifying relevant sources to user keyword queries and
directing the keywords to appropriate sources, which we call {\em keyword query routing}, will be a
key to keyword search on the data web.

\section{Contributions of This Thesis}\label{intro:contribution}\index{cotribution}

%content

\section{Organization}\label{intro:organization}\index{organization}

%Content