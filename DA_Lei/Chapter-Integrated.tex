\chapter{Keyword Query Routing for Data Web Search}\label{qr}

In this chapter, we propose a novel approach to address the problem of keyword query routing for
efficient keyword search on the data web. In Section~\ref{qr:problem}, we introduce the problem of
keyword query routing in the data web search setting. Section~\ref{qr:preprocessing} describes the
offline process of our approach where the data are preprocessed to construct the index. We present
the metrics to rank the keyword routing plans in Section~\ref{qr:ranking}. At last,
Section~\ref{qr:queryprocess} presents the query processing algorithm.

\section{Problem Definition}\label{qr:problem}

As discussed in Section~\ref{intro:motivation}, identifying the most relevant data sources to user
keyword queries and forwarding the keywords to appropriate sources will be a key to our data web
search setting. We observe that in this scenario the interconnected structures as answers might span
over different data sources, where each of them contains partial keywords of a query. To formally
describe the problem of keyword query routing, we first present several concepts as follows. At the
beginning, to facilitate further discussion of our approach for keyword query routing, we give the
definitions of {\em integrated data graph} and {\em integrated schema graph}:
\begin{definition}[Integrated Data Graph] An \emph{integrated data graph} $G_{ID}$ is a tuple
$(V,L,E)$ where
\begin{itemize}
  \item $V$ is a finite set of \emph{vertices}. Thereby, $V$ is conceived as the disjoint union
  $V_E\uplus V_C$ with E-vertices $V_E$ (representing entities) and C-vertices $V_C$ (classes),
  where each vertex belongs to a unique data source. 
  \item $L$ is a finite set of \emph{edge labels}, subdivided by $L = L_R \uplus \{mapping\} \uplus
  \{type\}$, where $L_R$ represents inter-entity relations, $\{mapping\}$ stands for entity
  or class mappings and $\{type\}$ capture the class membership of an entity.
  \item $E$ is a finite set of \emph{edges} of the form $e(v_1,v_2)$ with $v_1,v_2\in V$ and $e\in
  L$, where:
  \begin{itemize}
    \item $e \in L_R$ if and only if $v_1, v_2 \in V_E$,
    \item $e = \mathit{mapping}$ if and only if $v_1, v_2 \in V_E$ or $v_1, v_2 \in V_C$.
    \item $e = \mathit{type}$ if and only if $v_1 \in V_E$ and $v_2 \in V_C$.  
  \end{itemize}
\end{itemize}
\end{definition}

Moreover, each E-vertex $v$ can be modeled as a keyword collection that contains a keyword $k$ when
(1) an attribute value or the name of an attribute of the corresponding entity of $v$ contains $k$,
(2) the name of the corresponding class of a C-vertex that connected with $v$ contains $k$, or
(3) the name of the corresponding relation of an edge $e(v_1,v_2)$ contains $k$ where $e \in L_R$
and $v = v_1$ or $v_2$.  

\begin{definition}[Integrated Schema Graph] An \emph{integrated schema graph} $G_{IS}$ of an
\emph{integrated data graph} $G_{ID} = (V,L,E)$ is a tuple $(V',L',E')$ with vertices $V' = V_C$,
edge labels $L' = L_R \uplus \{mapping\}$ and edges $E'$ of the form $e(v_1,v_2)$ with $v_1, v_2 \in
V'$ and $e \in L'$. In particular, every vertex $v' \in V'$ represents an aggregation of all the
vertices $v \in V$ having the type $v'$, i.e., $v$ and $v'$ are connected by an type edge as
$type(v,v')$ in $G_{ID}$. Accordingly, we have $e(v'_1,v'_2)\in E'$ if and only if there is an edge
$e(v_1,v_2)\in E$ for some $v_1$ and $v_2$ that have the type $v'_1$ and $v'_2$, respectively.
\end{definition}

It is obvious that each C-vertex can be seen as a set of E-vertices, where every E-vertex represents
an entity. For a C-vertex $v'$ and an E-vertex $v$, we use $v \in v'$ to denote that $v$ have the
type $v'$, i.e., there exist an edge $type(v,v')$ in the integrated data graph $G_{ID}$. In the rest
of this thesis, C-vertex(E-vertex) will be interchangeably employed instead of entity set(entity) if
there is no ambiguity.

\begin{definition}[Keyword Routing Plan] Given a set of data sources $\mathcal{DS} =
\{ds_1,ds_2,\ldots,ds_{\left\vert\mathcal{DS}\right\vert}\}$ that together with mappings relating
them form the data web and a keyword query $\mathcal{K} =
\{k_1,k_2,\ldots,k_{\left\vert\mathcal{K}\right\vert}\}$, a mapping $\mu: \mathcal{K} \to
\mathcal{DS}$ from all keywords in the query to the data sources will be called a \emph{keyword
routing plan} $\mathcal{RP}$.
\end{definition}

Given a keyword query, we would like to find and rank the corresponding keyword routing plans based
on their relevance to the query. Ideally, the relevance of a keyword routing plan to a given query
can be measured as the total score of the results that can be generated by this keyword routing plan.
However, in practice, it is not feasible to get the total score of results for each keyword routing
plan, since it needs to execute the query over the whole data web. Thereby, it is necessary to put
forward an efficient way to address this problem. In the following, we propose a novel method that
summarizes the keyword relationships for efficiently finding the relevant keyword routing plans in
the data web search scenario and develop an effective ranking mechanism to rank them.

\begin{definition}[Candidate Entity Set Graph]\label{CG} Given a keyword query $\mathcal{K} =
\{k_1,k_2,\ldots,k_{\left\vert\mathcal{K}\right\vert}\}$, a \emph{candidate entity set graph}
$\mathcal{CG} = (V_{\mathcal{CG}},E_{\mathcal{CG}})$ is a complete graph with
$\left\vert\mathcal{K}\right\vert$ vertices. For each query keyword $k_i \in \mathcal{K}$, there is a
vertex $v''_i \in V_{\mathcal{CG}}$ represented as a combination of the query keyword and a C-vertex
$\langle k_i,v'_i \rangle$, which denotes that at least one entity in the entity set $v'_i$ contains
the keyword $k_i$. For each pair of vertices $v''_i(\langle k_i,v'_i \rangle)$ and $v''_j(\langle
k_j,v'_j \rangle)$, there exists an edge $(v''_i,v''_j) \in E_{\mathcal{CG}}$ associated with the
corresponding distances, which indicates that for every distance $dis$, there exists at least two
entities that belong to entity set $v'_i$ and $v'_i$ and contain keywords $k_i$ and $k_j$,
respectively, and they can be connected by a path at distance $dis$.
\end{definition}

It is clear that each candidate entity set graph can contribute to a unique keyword routing plan.
For a candidate entity set graph $\mathcal{CG}$ and a keyword routing plan $\mathcal{RP}$, we use
$\mathcal{CG} \in \mathcal{RP}$ to denote that candidate entity graph $\mathcal{CG}$ contributes to
the keyword routing plan $\mathcal{RP}$. Therefore, in our approach, given a keyword query, we first
achieve the relevant candidate entity set graphs associated with the relevance scores and then
purchase the final keyword query routing plans by combining such candidate entity set graphs that
contribute to the same plan.

\begin{definition}[Candidate Entity Set Connection] Given a keyword pair $\langle k_1,k_2\rangle$, a
\emph{candidate entity set connection} $\mathcal{CC}$ of this two keywords combination is a pair of
C-vertices associated with corresponding distances of the form $\langle
v'_1,v'_2,(dis_1,\ldots,dis_n)\rangle$, where for each distance $dis_i$ there exist at least a path
at distance $dis_i$ form an E-vertex $v_1$, which has the type $v'_1$ and contains the keyword $k_1$,
to an E-vertex $v_2$, which has the type $v'_2$ and contains the keyword $k_2$, in the integrated
data graph $G_{ID}$.
\end{definition}

The candidate entity set connections for each keyword pair can be computed offline during data
preprocessing discussed in Section~\ref{qr:preprocessing} and then it can be employed to construct
the candidate entity set graphs during query processing discussed in Section~\ref{qr:queryprocess}.

We now formally state the problem of keyword query routing in the data web search scenario. 
\begin{problem}
Given an integrated data graph $G_{ID}$ and a keyword query $\mathcal{K}$, the problem of keyword
query routing is to find the top-$k$ keyword routing plans based on their relevance to the keyword
query $\mathcal{K}$.
\end{problem}

\section{Data Preprocessing}\label{qr:preprocessing}

In order to perform query processing for producing the top-$k$ keyword routing plans in an efficient
way, we preprocess the underlying data to compute the candidate entity set connections for all
keyword pairs and devise an extended inverted index to maintain them.

For this offline indexing process, three steps need to be performed. Initially, we compute the
relationships between entities in the integrated data graph at various distances. In the second step,
based on the relationships of entities at different distances, we extract the connected term pairs
associated with the entities containing them. Finally, given the obtained term pairs, the associated
entities are clustered based on an aggregation rule to construct the candidate entity set connections
and then they will be stored in an extended inverted index. The following subsections will describe
these three steps of data preprocessing respectively. 

\subsection{Entity Relationships Computation}

For entity relationships computations, we model the connections of entities at various distances $d$
as $n\times n$ symmetric boolean matrices $\mathcal{R}(d)$  with respect to integrated data graph
$G_{ID}$, where $n$ is the total number of E-vertices in $G_{ID}$, in other words, $n$ is the total
number of entities in all data sources. In $\mathcal{R}(d)$, the element $r^d_{ij}$ is 1, iff, there
is an path between $v_i$ and $v_j$ at distance $d$ composed of the edges that represent inter-entity
relation or entity mapping in $G_{ID}$, where $v_i$($v_j$) denotes the E-vertex at i-th(j-th) row or
column in $\mathcal{R}(d)$; otherwise, $r^d_{ij}$ is 0. In general, users are interested in the
compact interconnected structures that contain the query keywords, thus a long path that connects two
keyword elements might make no sense for the information need of the users. We provide a threshold
$d_{max}$ to constrain the range of the distance and only the relationships of entities at distance
$d \leq d_{max}$ are considered. The matrix $\mathcal{R}(1)$, illustrated as,
\[\mathcal{R}(1) = (r^1_{ij})_{n \times n} = \begin{Bmatrix} 
     \quad & \quad & v_1 & v_2 & \cdots & v_n & \quad\\ 
     \quad & v_1 & 0 & 1 & \cdots & 1 & \quad\\ 
     \quad & v_2 & 1 & 0 & \cdots & 0 & \quad\\ 
     \quad & \vdots & \vdots & \quad \\
     \quad & v_n & 1 & 0 & \cdots & 0 & \quad\\ 
\end{Bmatrix}\]
represents the direct connection of every E-vertex pair by an edge denoting inter-entity relation or
entity mapping in $G_{ID}$. If there exist an edge $e(v_i,v_j)$ in $G_{ID}$, the corresponding entry
$r^1_{ij}$ is 1; otherwise, $r^1_{ij}$ is 0. For $1 < d \leq d_{max}$, the matrix $\mathcal{R}(d)$
can be constructed iteratively using the following formula, 
\[\mathcal{R}(d) = (r^d_{ij})_{n \times n} = \mathcal{R}(d-1) \times \mathcal{R}(1)\] which
represents the connection of every E-vertex pair through a path at distance $d$ in $G_{ID}$.

\subsection{Connected Terms and Entity Connections Extraction}

In this step, we first extract the terms of all E-vertices and then a lexical analysis (stemming,
removal of stop words) will be performed on them. Note that the E-vertices contain two types of
terms. The first type is terms that are matched with attribute values, and we call them \emph{value
terms}. The second type is terms that are matched with the name of class, attributes and relations,
and we call them \emph{schema terms}. For the \emph{schema terms}, a semantical analysis are
performed on them, where semantically similar entries such as synonyms, hyponyms and hypernyms are
extracted from WordNet\footnote{\url{http://wordnet.princeton.edu/}} \citep{wordnet} and
incorporated in the corresponding E-vertices.

After that, based on the relationships of entities at various distances, all term pairs contained
either in the connected entities or the same entities will be extracted. For each pair of terms
$\langle t_i,t_j \rangle$, if there exist two E-vertices $v_i$ and $v_j$ that contain the $k_i$ and
$k_j$ respectively and the entry $r^d_{ij}$ of $\mathcal{R}(d)$ is 1, we create a tuple $\langle
v_i,v_j,d \rangle$ that we call an \emph{entity connection} of $\langle t_i,t_j \rangle$. In
particular, for the pair of terms $\langle t_i,t_j \rangle$ that can be contained in the same
E-vertex $v$, the \emph{entity connection} $\langle v,v,0 \rangle$ will be created. Such entity
connections are linked with the references of corresponding term pairs.

Ideally, given a keyword query $\mathcal{K}
=\{k_1,k_2,\ldots,k_{\left\vert\mathcal{K}\right\vert}\}$, we can retrieve the entity connections
$\langle v_i,v_j,d \rangle$ for each keyword pair $\langle k_i,k_j \rangle$, and then find the
\emph{candidate entity graphs} based on the obtained entity connections. In the \emph{candidate
entity graph}, every vertex can be represented as an E-vertex $v_i$ containing one keyword $k_i$ of
the given query. Furthermore, the \emph{candidate entity graph} is complete, i.e., between each pair
of E-vertices $\langle v_i,v_j \rangle$, there is an edge which indicates that $v_i$ and $v_j$ can be
connected by a path or they are the same vertex in the integrated data graph $G_{ID}$. Obviously,
every \emph{candidate entity graph} might correspond to one or more answers of the keyword search on
the data web. In particular, if the intermediate elements, namely the vertices and edges in the
integrated data graph $G_{ID}$, between the vertices in the \emph{candidate entity graphs} forming
the connecting paths are figured out, then the interconnected structures as the answers of the
keyword query can be constructed. Hence, in this way, the keyword routing plans can be achieved from
such \emph{candidate entity graphs}. However, it might be too expensive to process the entity
connections for finding the \emph{candidate entity graphs} due to the large size of entities in the
integrated data graph $G_{ID}$ and accordingly perhaps a large number of entity connections for some
keyword pairs. Therefore, in our approach, such entity connections will be clustered to get the
entity set connections discussed in the next step.

\subsection{Entity Set Connections Construction and Indexing}

In the final phase of data preprocessing, the obtained entity connections in the previous phase are
clustered to get the entity set connections and then such entity set connections will be stored in
the index.

For each term pair $\langle t_i,t_j \rangle$, the corresponding entity connections $\langle
v_i,v_j,dis \rangle$ are aggregated to the entity set connections $\langle
v'_i,v'_j,(dis_1,\ldots,dis_n) \rangle$, when E-vertices $v_i$ and $v_j$ are connected with
C-vertices $v'_i$ and $v'_j$ by $type$ edge in the integrated data graph $G_{ID}$, respectively. In
addition, the distance $dis$ in the entity connections $\langle v_i,v_j,dis \rangle$ will be added
into the corresponding entity set connections $\langle v'_i,v'_j,(dis_1,\ldots,dis_n) \rangle$. For
each distance, we track and store the number of entity connections that collapsed during the
clustering. We call such number \emph{entity connection frequency} that will be used for scoring.
Note that if there is no C-vertex that is connected with E-vertex $v_i$(or $v_j$) in the integrated
data graph $G_{ID}$ and given that $v_i$(or $v_j$) belongs to the data source $ds_i$(or $ds_j$), the
corresponding $v'_i$(or $v'_j$) will be assigned as a special C-vertex $v^*_{ds_i}$(or $v^*_{ds_j}$)
called \emph{top class} of data source $ds_i$(or $ds_j$)  that represents the aggregation of all the
E-vertices in the data source $ds_i$(or $ds_j$) with no given type.

We employ an extended inverted index, as depicted in Table \ref{index}, to maintain the entity set
connections, where the entries of such index are term pairs(combinations of two keywords) and the
values of each entry are corresponding entity set connections and the associated scores. The
computation of such scores will be discussed in Section \ref{qr:ranking}. Given a keyword query
$\mathcal{K} = \{k_1,k_2,\ldots,k_{\left\vert\mathcal{K}\right\vert}\}$, we first retrieve the entity
set connections for each keyword pair $\langle k_i,k_j \rangle$ according to the index, and then
based on them the candidate entity set graphs are constructed which will be discussed in
Section~\ref{qr:queryprocess}. Finally, the top-$k$ keyword routing plans can be achieved from such
candidate entity set graphs associated with scores.

{\tiny
\begin{table}[tbp]
\centering
\begin{tabular}{|l||l|}
\hline
\textbf{Term Pair} & \textbf{Indexed Structure} \\
\hline
\hline
$\langle term_i,term_j \rangle$  & [$\langle v'_i,v'_j,(dis_1,\ldots,dis_n) \rangle,score$] \\
\hline
\end{tabular}
\caption{Indexing Entity Set Connections} \label{index}
\end{table}
}

\section{Ranking Functions}\label{qr:ranking}

Since we are interested in only the top-$k$ keyword query routing plans, in this section, we propose
a novel ranking mechanism for ranking keyword routing plans based on some scoring functions. The
scoring of answers with respect to keyword search on the structured data has been extensively
discussed, see \citep{DBLP:conf/icde/BhalotiaHNCS02, DBLP:conf/sigmod/HeWYY07,
DBLP:conf/vldb/HristidisP02, DBLP:conf/vldb/HristidisGP03, DBLP:conf/sigmod/LiuYMC06,
DBLP:conf/sigmod/LuoLWZ07, DBLP:conf/icde/TranWRC09}. In the following, we introduce several new
scoring functions that aim to assess the relevance of the keyword routing plans to the given keyword
query.

The basic idea of our ranking method is that we first assign each candidate entity set graph a score
and then the score of a keyword routing plan will be computed from the individual scores of candidate
entity set graphs that contribute to this keyword routing plan by using a score aggregation function.
For scoring the candidate entity set graphs, the relevance of both vertices and edges of such graphs,
in regard to keyword coverage and structure coverage respectively, will be taken into account. As is
well known, the \mbox{TF-IDF} model has proven to be an effective measure for ranking textual
documents in the context of document retrieval. By adopting this notion, we define some new metrics
to measure the relevance of vertices and edges of the candidate entity set graphs with respect to the
given keyword query.

\subsection{Scoring of Graph Vertices}

Analogous to term frequency $tf$ and inverse document frequency $idf$ in the \mbox{TF-IDF} model for
ranking documents, we define the entity frequency $ef$ and inverse entity set frequency $icf$ for the
given keyword $k_i$, where $ef_{k_i}$ denotes the frequency of entities that contain $k_i$ in a
entity set and $icf_{k_i}$, as illustrated in Equation \ref{icf},  can be seen as a measure for
distinctiveness of the keyword $k_i$ with respect to the entity sets. $N_C$ and $cf_{k_i}$ in
Equation \ref{icf} respectively denote the number of entity sets on the data web and the number of
entity sets that $k_i$ occurs.

\begin{equation}\label{icf}
icf_{k_i} = \ln\frac{N_C+1}{cf_{k_i}}
\end{equation}

We combine the two parameters $ef$ and $icf$ to evaluate the relevance of each vertex $v'_i$
corresponding to the query keyword $k_i$ in a given candidate entity set graph $\mathcal{CG}$,
denoted as $\textsc{Score}_{V}(k_i,\mathcal{CG})$, as formalized in Equation \ref{SVertex}. As noted
in Definition \ref{CG}, for each query keyword $k_i$, the corresponding vertex $v'_i$ in the
candidate entity set graph $\mathcal{CG}$ represents an entity set in which at least one entity
contains the $k_i$. In Equation \ref{SVertex}, we use $ef_{k_i}$ to denote the frequency of entities
that contain the keyword $k_i$, in other words, the number of entities where the keyword $k_i$
occurs, in $\mathcal{CG}$.

\begin{equation}\label{SVertex}
\textsc{Score}_{V}(k_i,\mathcal{CG}) =
ef_{k_i} * icf_{k_i} 
% = ef_{k_i} * \ln\frac{N_C+1}{cf_{k_i}}
\end{equation}

Based on the relevance of each vertex $v'_i$ corresponding to the query keyword $k_i$ in
$\mathcal{CG}$, we compute the overall score of the vertices in $\mathcal{CG}$ by summing up
$\textsc{Score}_{V}(k_i,\mathcal{CG})$, as shown in Equation \ref{SVertices}.

\begin{equation}\label{SVertices}
\textsc{Score}_{V}(\mathcal{K},\mathcal{CG}) =
\sum_{k_i \in \mathcal{K}} \textsc{Score}_{V}(k_i,\mathcal{CG})
\end{equation}

$\textsc{Score}_{V}(\mathcal{K},\mathcal{CG})$ measures the overall relevance between the keyword
query $\mathcal{K}$ and the candidate entity set graph $\mathcal{CG}$ in respect of keyword coverage.
However, it is not sufficient to consider only the factor of keyword coverage, because the answers
of keyword search on the data web involve rich structural relationships, which are even more important
in some cases. Therefore, it is essential to incorporate the scoring functions in respect of
structure relationship.
 
\subsection{Scoring of Graph Edges}

In the similar way to scoring of graph vertices, we define the entity connection frequency $ecf$ and
inverse entity set connection frequency $iccf$ for the given keyword pair $\langle k_i,k_j \rangle$,
where $ecf_{\langle k_i,k_j \rangle}$ denotes the frequency of entity connections for $\langle
k_i,k_j \rangle$ in a entity set connection and $iccf_{\langle k_i,k_j \rangle}$, as illustrated in
Equation \ref{iccf}, can be seen as a measure for distinctiveness of the connected keyword pair
$\langle k_i,k_j \rangle$ with respect to the entity set connections. $N_{CC}$ and $ccf_{\langle
k_i,k_j \rangle}$ in Equation \ref{iccf} respectively denote the number of entity set connections for
all connected keyword pairs on the data web and the number of entity set connections for the keyword
pair $\langle k_i,k_j \rangle$.

\begin{equation}\label{iccf}
iccf_{\langle k_i,k_j \rangle} = \ln\frac{N_{CC}+1}{ccf_{\langle k_i,k_j \rangle}}
\end{equation}

Furthermore, it is noted that, for each entity connection $\langle v_i,v_j,d \rangle$, where $d$
represents the length of the path connecting $v_i$ and $v_j$, when $d$ is larger, the structural
relevance of $v_i$ and $v_j$ is smaller. Therefore, we introduce a function of $d$  $\varphi_d$ that
measures the impact of $d$ to scoring of the candidate entity set graph edges. It is clear that
$\varphi_d$ should be a monotonically decreasing function with respect to increasing $d$. In our
approach, we define $\varphi_d$ in Equation \ref{fd}.

\begin{equation}\label{fd} 
\varphi_d = \frac{1}{d+1}
\end{equation}

We combine the three parameters $ecf$, $iccf$ and $\varphi_d$ to evaluate the relevance of each edge
$(v'_i,v'_j)$ corresponding to the query keyword pair $\langle k_i,k_j \rangle$ in a given candidate
entity set graph $\mathcal{CG}$, denoted as $\textsc{Score}_{E}(\langle k_i,k_j
\rangle,\mathcal{CG})$, as formalized in Equation \ref{SEdge}. As noted in Definition \ref{CG}, for
each query keyword pair $\langle k_i,k_j \rangle$, the corresponding edge $(v'_i,v'_j)$ in the
candidate entity set graph $\mathcal{CG}$ represents an entity set connection $\langle
v'_i,v'_j,(d_1,\ldots,d_n)$ in which for each distance $d_i$ there exists at least one entity
connection $\langle v_i,v_j,d_i \rangle$ where $v_i$ and $v_j$ belong to entity set $v'_i$ and $v'_j$
and contain keywords $k_i$ and $k_j$, respectively, and they can be connected by a path at distance
$d_i$. In Equation \ref{SEdge}, we use $ecf_{\langle k_i,k_j \rangle}(d)$ to denote the frequency of
entity connections, namely the number of entity connections, of distance $d$ for the keyword pair
$\langle k_i,k_j \rangle$ in $\mathcal{CG}$.

\begin{eqnarray}\label{SEdge}
\textsc{Score}_{E}(\langle k_i,k_j \rangle,\mathcal{CG}) =
\sum_{d}\varphi_d * ecf_{\langle k_i,k_j \rangle}(d) * iccf_{\langle k_i,k_j \rangle}
% \nonumber\\ =
% \sum_{d}\frac{1}{d+1} * ecf_{\langle k_i,k_j \rangle}(d) * \ln\frac{N_{CC}+1}{ccf_{\langle
% k_i,k_j \rangle}}
\end{eqnarray}

Based on the relevance of each edge $(v'_i,v'_j)$ corresponding to the query keyword pair $\langle
k_i;k_j \rangle$ in $\mathcal{CG}$, we compute the overall score of the edges in $\mathcal{CG}$ by
summing up $\textsc{Score}_{E}(\langle k_i,k_j \rangle,\mathcal{CG})$, as shown in Equation
\ref{SEdges}.

\begin{equation}\label{SEdges}
\textsc{Score}_{E}(\mathcal{K},\mathcal{CG}) =
\sum_{\{k_i,k_j\} \subseteq \mathcal{K},i\neq j} \textsc{Score}_{E}(\langle k_i,k_j
\rangle,\mathcal{CG})
\end{equation}

$\textsc{Score}_{E}(\mathcal{K},\mathcal{CG})$ measures the overall relevance between the keyword
query $\mathcal{K}$ and the candidate entity set graph $\mathcal{CG}$ in respect of structure
relationship.

\subsection{Scoring of Keyword Routing Plan}

% \begin{equation}\label{}
% \textsc{Score}(\mathcal{K},\mathcal{CG}) = \textsc{Score}_{V}(\mathcal{K},\mathcal{CG}) *
% \textsc{Score}_{E}(\mathcal{K},\mathcal{CG})
% \end{equation}

By taking into consideration the relevance with respect to both the keyword coverage and the
structure relationship, we integrate the two scores, $\textsc{Score}_{V}(\mathcal{K},\mathcal{CG})$
(Equation \ref{SVertices}) and $\textsc{Score}_{E}(\mathcal{K},\mathcal{CG})$ (Equation \ref{SEdges})
to score the overall relevance between the keyword query $\mathcal{K}$ and the candidate entity set
graph $\mathcal{CG}$, denoted as $\textsc{Score}(\mathcal{K},\mathcal{CG})$, as described in
Equation \ref{SGraph}, where $\alpha$ and $\beta$ are tunable weighting parameters.

\begin{equation}\label{SGraph}
\textsc{Score}(\mathcal{K},\mathcal{CG}) = \alpha\textsc{Score}_{V}(\mathcal{K},\mathcal{CG}) +
\beta\textsc{Score}_{E}(\mathcal{K},\mathcal{CG})
\end{equation}

Thereafter, the score of a keyword routing plan can be computed from the individual scores of all
candidate entity set graphs that contribute to this keyword routing plan by using a score aggregation
function, $\mathtt{SUM}$ in our approach. Additionally, given a keyword query, we observe that the
answers which span over less data sources are more likely match the information need of the users due
to the uncertainty of the mappings by which the data sources are interlinked. Based on this
observation, we assume that the keyword routing plans involved in less data sources are preferred.
Therefore, given a keyword routing plan $\mathcal{RP}$, we introduce a function
$\kappa_{\mathcal{RP}}$ with respect to the number of data sources that $\mathcal{RP}$ involved in,
denoted as $\left\vert{DS}_{\mathcal{RP}}\right\vert$, to scoring of the keyword routing plan
$\mathcal{RP}$. We propose to set $\kappa_{\mathcal{RP}}$ as Equation \ref{nds}.

\begin{equation}\label{nds}
\kappa_{\mathcal{RP}} = \frac{1}{\left\vert{DS}_{\mathcal{RP}}\right\vert}
\end{equation}

Finally, we present the function for scoring keyword routing plan $\mathcal{RP}$ with respect to
keyword query $\mathcal{K}$ as given in Equation \ref{SPlan}.

\begin{equation}\label{SPlan}
\textsc{Score}(\mathcal{K},\mathcal{RP}) =
\kappa_{\mathcal{RP}} * \sum_{\mathcal{CG} \in \mathcal{RP}}\textsc{Score}(\mathcal{K},\mathcal{CG})
\end{equation}

\section{Query Processing}\label{qr:queryprocess}

\begin{algorithm}[tb]\label{alg:queryplan}
 	\SetLine \linesnumbered \caption{Candidate Entity Set Graph Computation}
 	\KwIn{$K = (k_1,\ldots,k_{\left\vert K \right\vert})$.}
	\KwData{$DC = (LC_1,\ldots,LC_{(\left\vert K \right\vert - 1)\left\vert K \right\vert/2})$, where
	$LC_i = (CC_{i1},\ldots,CC_{i\left\vert LC_i \right\vert})$ with corresponding keyword pair as
	$\langle k_{i1},k_{i2}\rangle$ and $CC_{ij}$ represents a candidate entity set connection of
	$\langle k_{i1},k_{i2}\rangle$; 
	$LC = (v_1,\ldots,v_{\left\vert LD \right\vert})$, where $v_i$ represents a C-vertex; 
	$LG = (CG_1,\ldots,CG_{\left\vert LG\right\vert})$ where $CG_i$ represents a candidate entity set
	graph.}

	\KwResult{set of candidate entity set graphs}
	
 	\SetKwFunction{retrieveComb}{retrieveCCs}
 	\SetKwFunction{minCombList}{minCCList}
 	\SetKwFunction{NextCandidateKeyword}{pickKeyword}
 	\SetKwFunction{NextCandidate}{EntitySetList}

 	\ForEach{keyword pair $\langle k_i,k_j\rangle$ where $k_i,k_j \in K$ and $i \neq j$} {
 		$DC$.add(\retrieveComb{$\langle k_i,k_j\rangle$})\;
 	}
 	
 	$LC \leftarrow$ \minCombList{$DC$}\;
	$K \leftarrow K \setminus \{LC.k_1,LC.k_2\}$\;
 	
 	\ForEach{$C_i \in LC$}{
		$CG \leftarrow (\langle LC.k_1,CC_i.v_1\rangle,\langle LC.k_2,CC_i.v_2\rangle)$\;
 		$LG$.add($CG$)\;
 	}

 	\While{$LG \neq \emptyset$ and $K \neq \emptyset$}{
		$k \leftarrow$ \NextCandidateKeyword{$K$}\;
	    $K \leftarrow K \setminus \{k\}$\;
		$LG' \leftarrow \emptyset$\; 
		\ForEach{$CG \in LG$}{
			\ForEach{$\langle k_i,v_i\rangle \in CG$}{
				$LC_i \leftarrow$ \NextCandidate{$k_i,k,v_i$}\;
			}
			$LC \leftarrow$ mergeAll($LC_i$)\;
			\ForEach{$v \in LC$}{
				$CG' \leftarrow CG$\;
				$CG'$.add($\langle k,v\rangle$)\;
				$LG'$.add($CG'$)\;
			}
		}
		$LG \leftarrow LG'$\;
 	}
	\Return $LG$\;
 
\end{algorithm}



