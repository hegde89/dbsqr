\chapter{Preliminary}\label{pre}

In this chapter we firstly formalize the keyword search problem on data web in
section~\ref{pre:ksdw}, where various models used throughout this thesis are discussed and defined.
Section~\ref{pre:routing} introduces the problem of keyword query routing tackled by this thesis.

\section{Problem Definition of Keyword Search on the Data Web}\label{pre:ksdw}

As mentioned in section~\ref{intro:motivation}, in the data web search setting, answering keyword
queries might involve multiple data sources. In addition, it is assumed that these data sources are
not independent but interlinked by mappings. We elaborate on the problem of keyword search
on data web in what follows. We will begin with a model for underlying data, and continue with the
query and answer models followed by the scoring model. At last we will introduce the problem of
keyword query routing for data web search based on the defined models.

\subsection{Data Model} 

We consider the data web as a set of interrelated web data sources, each of them identified using a
data source identifier. We use a graph-based data model to characterize individual web data sources.
Since the structured data on the web varies greatly in their forms, such as XML data, relational data
and RDF\footnote{\url{http://www.w3.org/RDF/}}, a framework for (Web) resource description
standardized by the W3C, we will present a general data model that is sufficiently to capture all of
them.

\begin{definition}\label{def:datagraph}
A \emph{data graph} $G$ is a tuple $(V,L,E)$ where
\begin{itemize}
  \item $V$ is a finite set of \emph{vertices}. Thereby, $V$ is
  conceived as the disjoint union $V_E\uplus
  V_C\uplus V_V$ with E-vertices $V_E$
  (representing entities), C-vertices $V_C$
  (classes), and V-vertices $V_V$ (attribute values).
  \item $L$ is a finite set of \emph{edge
  labels}, subdivided by $L = L_R \uplus L_A \uplus
  \{type\}$, where $L_R$ represents
  inter-entity relations and $L_A$ stands for
  entity-attribute assignments. \item $E$ is a
  finite set of \emph{edges} of the form
  $e(v_1,v_2)$ with $v_1,v_2\in V$ and $e\in L$.
  Moreover, the following restrictions apply:
  \begin{itemize}
    \item $e \in L_R$ if and only if $v_1, v_2 \in V_E$,
    \item $e \in L_A$ if and only if $v_1 \in V_E$ and $v_2 \in V_V$, and
    \item $e = \mathit{type}$ if and only if $v_1 \in V_E$ and $v_2 \in V_C$. 
  \end{itemize}
\end{itemize}
\end{definition}

To interrelate the elements of individual data sources, our data model is extended with mappings:

\begin{definition} A \emph{mapping} $M$ is set of mapping assertions representing approximate
correspondences between graph elements. Specifically, mapping assertions in $M$ are of the form
$m(v_1,v_2, s)$ where $v_1,v_2 \in V$ are graph vertices  and $s \in [0,1]$ is a score denoting a
confidence value associated with the mapping.
\end{definition}

Data sources together with mappings relating them form a \emph{data web} defined as an {\em
integrated data graph}:

\begin{definition}\label{def:idatagraph} An \emph{integrated data graph} $g_{ID}$ is a tuple
$(G_D,M_D)$, where $G_D$ is a finite set of data graphs and $M_D$ is a set of approximate
correspondences between data elements called \emph{entity mappings}.
\end{definition}

While edges of a particular graph are called \emph{intra-data-source edges}, edges representing
mappings between elements of different data graphs will be referred to as \emph{inter-data-sources
edges}.


\subsection{Query and Answer Model}
In this data web scenario, a keyword query is given as a set of keywords $\mathcal{K} =
\{k_1,k_2,\ldots,k_{\left\vert\mathcal{K}\right\vert}\}$, and is to search the interconnected
structures that contain the given keywords in the integrate data graph, where the keywords might
refer to C-vertices (classes), V-vertices (data values) and edges. In brief, it figures out how the
elements containing the given keywords are connected via sequences of connections.

In the literature, in terms of the interconnected structure as answers there are some different
types, such as minimal total joined trees up to a certain size\citep{DBLP:conf/icde/AgrawalCD02,
DBLP:conf/vldb/HristidisP02, DBLP:conf/vldb/HristidisGP03, DBLP:conf/sigmod/LiuYMC06,
DBLP:conf/sigmod/LuoLWZ07, DBLP:conf/icde/SayyadianLDG07}, subtrees derived from a distinct root
\citep{DBLP:conf/icde/BhalotiaHNCS02, DBLP:conf/vldb/KacholiaPCSDK05, DBLP:conf/sigmod/HeWYY07},
subgraphs derived from a connecting element within a radius \citep{DBLP:conf/icde/TranWRC09} or
multi-center subgraphs within a radius \citep{DBLP:conf/icde/QinYCT09}. All these distinct
interconnected structures represent different semantics and are needed for different applications.
Accordingly, to achieve these different semantics the data are dealt in different ways.

\section{Problem Definition of Keyword Query Routing}\label{pre:routing}

As discussed in section~\ref{intro:motivation}, identifying the most relevant data sources to user
keyword queries and directing the keywords to appropriate sources will be a key to our data web
search setting. We observe that in this scenario the interconnected structures as answers might span
over different data sources, where each of them contains partial keywords of a query. To formally
describe the problem of keyword query routing, we first present the concept of {\em query routing
plan} as follows.

\begin{definition}
Given a set of data sources $\mathcal{DS} =
\{ds_1,ds_2,\ldots,ds_{\left\vert\mathcal{DS}\right\vert}\}$ that together with mappings relating
them form the data web and a keyword query $\mathcal{K} =
\{k_1,k_2,\ldots,k_{\left\vert\mathcal{K}\right\vert}\}$, a mapping $\mu: \mathcal{K} \to
\mathcal{DS}$ from all keywords in the query to the data sources will be called a \emph{query
routing plan}.
\end{definition}

Intuitively, if a query routing plan can generate more answers with higher relevance scores, we
consider it as more relevant. Thereby, the problem addressed in this thesis is to find the top-k
query routing plans for efficient keyword search on the data web.

To faciliate further discussion of our approach for keyword query routing, we give a modified
definition\footnote{This modified model coincides with Definition \ref{def:idatagraph}, which is
employed for facilitation of dicussion.} of {\em integrated data graph}:

\begin{definition}\label{def:idatagraph2}
A \emph{integrated data graph} $g_{ID}$ is a tuple $(V_E,L,E)$ where
\begin{itemize}
  \item $V_E$ is a finite set of E-vertices (representing entities), where each E-vertices is
  modeled as a keyword collection that contains a keyword when
  \begin{itemize}
    \item an attribute value of the corresponding entity contains the keyword,
    \item the name of a class that is type of the corresponding entity contains the keyword, or
    \item the name of a relation or an attributes that is connected with the corresponding entity
    contains the keyword.
  \end{itemize}
  \item $L$ is a finite set of \emph{edge
  labels}, subdivided by $L = L_R \uplus L_M$, where $L_R$ represents
  inter-entity relations and $L_M$ stands for entity mappings.
  \item $E$ is a
  finite set of \emph{edges} of the form $e(v_1,v_2)$ with $v_1,v_2\in V_E$ and $e\in L$.
\end{itemize}
\end{definition}



