\chapter{Preliminary}\label{pre}

\section{Keyword Search on Structured and Semi-Structured Data}

\subsection{Data Model} 

\subsection{Query and Answer Model}
A keyword query is given as a set of keywords $\mathcal{K} =
\{k_1,k_2,\ldots,k_{\left\vert\mathcal{K}\right\vert}\}$, and is to search the interconnected
structures that contain the given keywords from the underlying data. In brief, it figures out how the
elements containing the given keywords are connected via sequences of connections.

In the literature, in terms of the interconnected structure as answers there are some different
types, such as minimal total joined trees up to a certain size\citep{DBLP:conf/icde/AgrawalCD02,
DBLP:conf/vldb/HristidisP02, DBLP:conf/vldb/HristidisGP03, DBLP:conf/sigmod/LiuYMC06,
DBLP:conf/sigmod/LuoLWZ07, DBLP:conf/icde/SayyadianLDG07}, subtrees derived from a distinct root
\citep{DBLP:conf/icde/BhalotiaHNCS02, DBLP:conf/vldb/KacholiaPCSDK05, DBLP:conf/sigmod/HeWYY07},
subgraphs derived from a connecting element within a radius \citep{DBLP:conf/icde/TranWRC09} or
multi-center subgraphs within a radius \citep{DBLP:conf/icde/QinYCT09}. All these distinct
interconnected structures represent different semantics and are needed for different applications.
Accordingly, to achieve these different semantics the data are dealt in different ways.

\section{Keyword Search on the Data Web}
In the data web search setting, answering keyword queries might involve multiple data sources. In
addition, it is assumed that these data sources are not independent but interlinked by mappings. We
consider the data web as a set of interrelated web data sources, each of them identified using a data
source identifier.




